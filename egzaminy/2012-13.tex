\documentclass[12pt, a4paper]{article}
\usepackage[polish]{babel}
\usepackage[T1]{fontenc}
\usepackage[utf8]{inputenc}
\usepackage{mathtools}
\usepackage{amsfonts,amsmath,amssymb,amsthm}
\usepackage{enumerate}
\usepackage{nopageno} % usuwa numery stron
\usepackage[margin=.5in]{geometry} % zmniejsza margines
\usepackage{fouriernc} % śmieszna czcionka
\usepackage{algpseudocode}

\newcommand{\MBBR}{\mathbb{R}}
\newcommand{\MBBZ}{\mathbb{Z}}
\newcommand{\MBBN}{\mathbb{N}}
\newcommand{\MBBC}{\mathbb{C}}
\newcommand{\MBBK}{\mathbb{K}}
\newcommand{\MBBQ}{\mathbb{Q}}
\newcommand{\MCALO}{\mathcal{O}}

\newcounter{zadanie}
\newcommand{\zadanie}{\addtocounter{zadanie}{1}\section*{Zadanie \arabic{zadanie}}}

\title{Egzamin 2012/13}
\author{Stanisław Bitner}
\date{\today}

\begin{document}
\maketitle
\zadanie{}
Zauważmy, że liczby z przedziału $[0,n^3]$ można łatwo posortować w czasie
liniowym. Wystarczy spojrzeć na nie jako na liczby w systemie $n$-arnym
i posortować leksykograficznie.

\subsection*{(a)}
Sortujemy krawędzie pionowe najpierw po $x$ potem stabilnie po mniejszym $y$.
Żeby sprawdzić, czy punkt $(x,y)$ leży na którejś z nich wystarczy wyszukać
binarnie przedziału krawędzi o tym samym $x$. Następnie szukamy binarnie
krawędzi zaczynającej się poniżej $y$ i sprawdzamy, czy zawiera ona w sobie ten
punkt. Sprawdzamy to także dla $4$ krawędzi bezpośrednio pod tą znalezioną.
Analogicznie dla krawędzi poziomych.

\subsection*{(b)}
Skalujemy współrzędne (osobno $x$ i $y$) na takie od $1$ do $2n+m$, gdzie $m$
to liczba punktów. Sortujemy krawędzie pionowe i wrzucamy je do kubełków
odpowiadającym ich $x$, przy czym, jeśli dwie krawędzi mają ten sam niższy $y$,
to wstawiamy tę dłuższą z nich. Punkty tak samo. Przechodzimy po wszystkich
kubełkach i gdy napotkamy jakiś punkt, to sprawdzamy 5 niżej zaczętych
krawędzi, czy aby go nie zawierają. Następnie wyrzucamy z kubełka (listy) ten
punkt. Analogicznie dla krawędzi poziomych.

\subsection*{(c)}
Skalujemy współrzędne jak poprzednio. Sortujemy punkty końcowe po $x$.
Przechodzimy po nich od lewej do prawej. Gdy napotykamy lewy punkt poziomej
krawędzi dodajemy $1$ do $t[y]$, gdy napotykamy prawy, to odejmujemy. Gdy
napotykamy pionową krawędź, to do globalnego wyniku dodajemy sumę na przedziale
$[y1,y2]$ w $t$. Sumę na przedziale obliczamy liniowo, ale długość krawędzi
jest nie większa niż $5$, więc jest to stały wkład.

\zadanie{}
% TODO:

\zadanie{}
Dwa trójkąty stykają się co najwyżej jednym wierzchołkiem, inaczej tworzyłyby
dwuspójną.

\subsection*{(a), (b)}
Tworzymy drzewo dwuspójnych. Kolorujemy liście takiego drzewa w dowolny sposób
i odcinamy je. Powtarzamy ten krok, dopóki graf nie jest pusty.
Złożoność: $\MCALO(n)$.

\subsection*{(c)}
Jeśli znamy liczbę trójkątów, to wynikiem jest $\frac{t}{3}$. Jeśli jej nie
znamy, to wyznaczamy ich liczbę algorytmem szukania dwuspójnych w grafie.
Działa, ponieważ z każdego trójkąta możemy w oczywisty sposób wziąć co najwyżej
jedną krawędź, a zawsze możemy, ponieważ z trójkątów będących liśćmi wybieramy
tę niezawierającą punktu artykulacji i odcinamy te trójkąty.
Złożoność: $\MCALO(1)$ lub $\MCALO(n)$.

\zadanie{}
Dodajemy do kopca te elementy. Budujemy minimalny pełny kopiec binarny je
zawierający i poprawiamy. Złożoność: $MCALO(k + \log^2{n})$.

\end{document}
