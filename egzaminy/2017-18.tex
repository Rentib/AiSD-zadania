\documentclass[12pt, a4paper]{article}
\usepackage[polish]{babel}
\usepackage[T1]{fontenc}
\usepackage[utf8]{inputenc}
\usepackage{mathtools}
\usepackage{amsfonts,amsmath,amssymb,amsthm}
\usepackage{enumerate}
\usepackage{nopageno} % usuwa numery stron
\usepackage[margin=.5in]{geometry} % zmniejsza margines
\usepackage{fouriernc} % śmieszna czcionka
\usepackage{algpseudocode}

\newcommand{\MBBR}{\mathbb{R}}
\newcommand{\MBBZ}{\mathbb{Z}}
\newcommand{\MBBN}{\mathbb{N}}
\newcommand{\MBBC}{\mathbb{C}}
\newcommand{\MBBK}{\mathbb{K}}
\newcommand{\MBBQ}{\mathbb{Q}}
\newcommand{\MCALO}{\mathcal{O}}

\newcounter{zadanie}
\newcommand{\zadanie}{\addtocounter{zadanie}{1}\section*{Zadanie \arabic{zadanie}}}

\title{Egzamin 2017/18}
\author{Stanisław Bitner}
\date{\today}

\begin{document}
\maketitle
\zadanie{}
Przechodzimy po wszystkich odcinkach i dodajemy punkty do nich należące na
listę. Odcinki są równoległe do osi układów współrzędnych, więc jest to bardzo
łatwe. Takich punktów jest dokładnie $11n$. Następnie sortujemy te punkty
leksykograficznie i zliczamy te, które występują co najmniej $2$ razy.
Złożoność: $\MCALO(n)$.

\zadanie{}
\subsection*{(a)}
% TODO:

\subsection*{(b)}
% TODO:

\subsection*{(c)}
Sortujemy kubełkowo przedziały po pierwszej współrzędnej. Zauważmy, że jeśli
$v$ jest odwiedzony, to $v-1$ może być odwiedzony w najgorszym przypadku w tej
samej liczbie ruchów. Zapisujemy informacje o wierzchołkach rzeczywiście odwiedzonych
oraz tych, do których możemy dojść w jednym ruchu.
\begin{algorithmic}

  \State $res \gets 0, vis \gets 1, can \gets 0$
  \For{$p\in J$}
    \If{$p.first \le vis$}
      \State $can \gets \max\{can, p.second\}$
    \Else
      \State $res \gets res+1$
      \State $vis \gets can$
    \EndIf
  \EndFor
  \If{$can > vis$}
    \State $res \gets res + 1$
  \EndIf

\end{algorithmic}
Warto podkreślić, że takie rozwiązanie zakłada spójność grafu.

\subsection*{(d)}
Jeden przedział tworzy oczywiście klikę. Przedziały $(a,b), (c,d)$ takie, że
mają co najmniej $2$ punkty wspólne oczywiście są w jednej dwuspójnej.
Sortujemy przedziały po pierwszej współrzędnej. Łączymy przedziały $i,i+1$,
które mają co najmniej $2$ punkty wspólne, lub $1$ z nich całkowicie zawiera
się w drugim. Wynikiem będzie liczba pozostałych przedziałów.

\zadanie{}
% TODO:

\zadanie{}
Wzbogacamy drzewo splay o następujące atrybuty
\begin{itemize}
  \item \textit{rozmiar} -- rozmiar poddrzewa
  \item \textit{pierwszy} -- wartość pierwszego elementu w poddrzewie
  \item \textit{ostatni} -- wartość pierwszego elementu w poddrzewie
  \item \textit{lazy} -- ile mamy dodać do synów
  \item \textit{inc} -- najdłuższy rosnący spójny podciąg poddrzewa
  \item \textit{pre} -- najdłuższy rosnący prefiks
  \item \textit{suf} -- najdłuższy rosnący sufiks
\end{itemize}
W drzewie będzie zachowany następujący niezmiennik: aktualna wartość będzie
równa sumie wartości wierzchołka i atrybutów $\mathit{lazy}$ wszystkich jego
przodków. Wszystkie atrybuty można w prosty sposób obliczać na podstawie
atrybutów lewego i prawego syna oraz wartości wierzchołka, zatem nie ma
problemu z rotacjami. Należy pamiętać, by podczas przeszukiwania spychać
wartość $lazy$ do synów. Operacje $\mathit{Ini, Insert, Delete}$ często
pojawiały się na ćwiczeniach. Operację $\mathit{Add}$ można wykonać funkcjami
$\mathit{Split, Join}$: najpierw wydzielamy przedział $[i,j]$, następnie
dodajemy do jego korzenia $e$ (do atrybutów $\mathit{lazy,pierwszy,ostatni}$
i wartości), potem z powrotem łączymy przedziały. Operacja $\mathit{MaxInc}$
polega po prostu na zwróceniu atrybutu $\mathit{inc}$ korzenia. Złożoności:
$\mathit{MaxInc}: \MCALO(1)$, pozostałe: $\MCALO(\log{n})$.

\end{document}
