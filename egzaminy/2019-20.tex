\documentclass[12pt, a4paper]{article}
\usepackage[polish]{babel}
\usepackage[T1]{fontenc}
\usepackage[utf8]{inputenc}
\usepackage{mathtools}
\usepackage{amsfonts,amsmath,amssymb,amsthm}
\usepackage{enumerate}
\usepackage{nopageno} % usuwa numery stron
\usepackage[margin=.5in]{geometry} % zmniejsza margines
\usepackage{fouriernc} % śmieszna czcionka
\usepackage{algpseudocode}

\newcommand{\MBBR}{\mathbb{R}}
\newcommand{\MBBZ}{\mathbb{Z}}
\newcommand{\MBBN}{\mathbb{N}}
\newcommand{\MBBC}{\mathbb{C}}
\newcommand{\MBBK}{\mathbb{K}}
\newcommand{\MBBQ}{\mathbb{Q}}
\newcommand{\MCALO}{\mathcal{O}}

\newcounter{zadanie}
\newcommand{\zadanie}{\addtocounter{zadanie}{1}\section*{Zadanie \arabic{zadanie}}}

\title{Egzamin 2019/20}
\author{Stanisław Bitner}
\date{\today}

\begin{document}
\maketitle
\zadanie{}
\subsection*{(a)}
Niech $p$ -- liczba permutacji spełniających $a<b \land c<d$. Zauważmy, że
6-permutacji takich, że $a<b$ jest $\frac{6!}{2}$, gdyż warunek jest spełniony
w połowie z wszystkich permutacji. Gdy spojrzymy na permutacje spełniające
$a<b$, to połowa z nich spełnia też $c<d$, zatem $p = \frac{6!}{4}$. Minimalna
liczba porównań to $\lceil\log_2{p}\rceil = 8$. \qed

\subsection*{(b)}
Porównujemy $e,f$, bez straty ogólności $e<f$. Sortujemy w $3$ ruchach większe
elementy par. Bez straty ogólności $b>d>f$. Wstawiamy wyszukiwaniem binarnym
$a$ do $d,e,f$ i następnie $c$ do $a,f,e$. Razem wykonujemy $1+3+2+2 = 8$
operacji.

\zadanie{}
\subsection*{(a)}
Dla DFS-drzewa największa wysokość to oczywiście $n-1$ ($C_n$). Nie można
osiągnąć większej, gdyż drzewo DFS ma $n-1$ krawędzi.\\
Dla BFS-drzewa jest to $\lfloor\frac{n}{2}\rfloor$. Między każdym $v \in V$,
a $1$ istnieją co najmniej dwie wierzchołkowo rozłączne ścieżki. Z zasady
szufladkowej Dirichleta musi istnieć ścieżka długości nie większej niż
$\lfloor\frac{n}{2}\rfloor$.

\subsection*{(b)}
Dla DFS-drzewa najmniejsza wynik, to $2$. Można to uzyskać dla grafu\\
$G = \langle1..n, \{1,2\}\cup\{\{1,v\},\{v,2\} : 3\le v\le n\}\rangle$. Żaby
było mniej, to $G$ musiałoby być gwiazdą, a gwiazda nie jest dwuspójna.\\
Dla BFS-drzewa jest to $1$ ($K_n$). Nie może być mniej, bo $n>2$ i trzeba dojść
do każdego wierzchołka.

\subsection*{(c)}
Wybieramy dowolny wierzchołek $s$. Odpalamy BFS. Jeśli istnieje krawędź $(v,u)$
taka, że $dist(s,v) \equiv_2 dist(s,u)$, to istnieje zadana marszruta. Jeśli
istnieje, to odpalamy BFS z $v$. Poszukiwaną marszrutą będzie $s
\rightsquigarrow v \rightsquigarrow s$ lub $s \rightsquigarrow u \rightarrow
v \rightsquigarrow s$ w zależności od ich parzystości.
Złożoność: $\MCALO(n)$.

\zadanie{}
Tworzymy ciąg $x'_i = \frac{x_i}{x_j}$, gdzie $j = \max\{j | j<i \land x_j \neq
0\} \cup \{1\}$. Analogicznie konstruujemy $y'_i$. \\
Wykonujemy algorytm KMP na ciągu $x' \cdot \# \cdot y'$ wyznaczając tablicę
$p$.\\
Ciąg $(y)$ jest podobny do $x$ w indeksie $i \iff p[i + n - 1] = n-1$.

\zadanie{}
% TODO:

\end{document}
