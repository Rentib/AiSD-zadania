\documentclass[12pt, a4paper]{article}
\usepackage[polish]{babel}
\usepackage[T1]{fontenc}
\usepackage[utf8]{inputenc}
\usepackage{mathtools}
\usepackage{amsfonts,amsmath,amssymb,amsthm}
\usepackage{enumerate}
\usepackage{nopageno} % usuwa numery stron
\usepackage[margin=.5in]{geometry} % zmniejsza margines
\usepackage{fouriernc} % śmieszna czcionka
\usepackage{algpseudocode}

\newcommand{\MBBR}{\mathbb{R}}
\newcommand{\MBBZ}{\mathbb{Z}}
\newcommand{\MBBN}{\mathbb{N}}
\newcommand{\MBBC}{\mathbb{C}}
\newcommand{\MBBK}{\mathbb{K}}
\newcommand{\MBBQ}{\mathbb{Q}}
\newcommand{\MCALO}{\mathcal{O}}

\newcounter{zadanie}
\newcommand{\zadanie}{\addtocounter{zadanie}{1}\section*{Zadanie \arabic{zadanie}}}

\title{Egzamin 2019/20}
\author{Stanisław Bitner}
\date{\today}

\begin{document}
\maketitle
\zadanie{}
\subsection*{(a)}
Niech $p$ -- liczba permutacji spełniających $a<b \land c<d$. Zauważmy, że
6-permutacji takich, że $a<b$ jest $\frac{6!}{2}$, gdyż warunek jest spełniony
w połowie z wszystkich permutacji. Gdy spojrzymy na permutacje spełniające
$a<b$, to połowa z nich spełnia też $c<d$, zatem $p = \frac{6!}{4}$. Minimalna
liczba porównań to $\lceil\log_2{p}\rceil = 8$. \qed

\subsection*{(b)}
Najprościej będzie scalić elementy $a,b,c,d$ wykonując $3$ porównania,
a następnie użyć wyszukiwania binarnego, aby wstawić elementy $e, f$, wykonując
kolejno $2$ i $3$ porównania. To daje razem $8$ porównań.

\zadanie{}
\subsection*{(a)}
Dla DFS-drzewa największa wysokość to oczywiście $n-1$ ($C_n$). Nie można
osiągnąć większej, gdyż drzewo DFS ma $n-1$ krawędzi.\\
Dla BFS-drzewa jest to $\lfloor\frac{n}{2}\rfloor$. Czemu? Należy przy tym podkreślić,
że jest to wynik w grafie bez krawędzi wtórnych.

\subsection*{(b)}
Dla DFS-drzewa najmniejsza wynik, to $\lfloor\frac{n}{2}\rfloor$. Czemu?\\
Dla BFS-drzewa jest to $1$ ($K_n$). Nie może być mniej, bo $n>2$ i trzeba dojść
do każdego wierzchołka.

\subsection*{(c)}
Aby sprawdzić, czy istnieje zamknięta marszruta o nieparzystej długości,
wystarczy sprawdzić, czy graf jest dwudzielny. Jeśli nie jest, to taka
marszruta musi istnieć.\\
Aby ją znaleźć, wystarczy użyć dowolnego algorytmu znajdującego nieparzysty
cykl, na przykład tego używającego kolorowania.

\newpage\zadanie{}
Przypuśćmy, że $n>1$, w przeciwnym przypadku, problem jest trywialny. Wystarczy
użyć lekko zmodyfikowanego algorytmu KMP.
\begin{algorithmic}

  \Function{Podobne}{$(x)_n,(y)_m$}
    \State $s \gets (x)_n + \infty + (y)_m$
    \State $p[n] = \{0..0\}$
    \For{$i=1..n-1$}
      \State $j \gets p[i-1]$
      \While{$j>0 \land \frac{s[i]}{s[j]} = \frac{s[i-1]}{s[j-1]}$}
        \State $j \gets p[j-1]$
      \EndWhile
      \If{$\frac{s[i]}{s[j]} = \frac{s[i-1]}{s[j-1]}$}
        \State $j \gets j + 1$
      \EndIf
      \State $p[i] \gets j$
    \EndFor
  \EndFunction

\end{algorithmic}
Jeżeli $p[i]$ jest równe $n$, to $i$ jest końcem jednego z podobnych spójnych podciągów.
Skoro chcemy wyznaczyć początki podciągów, to wystarczy odejmować $n-1$ od tych indeksów.

\end{document}
