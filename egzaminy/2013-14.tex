\documentclass[12pt, a4paper]{article}
\usepackage[polish]{babel}
\usepackage[T1]{fontenc}
\usepackage[utf8]{inputenc}
\usepackage{mathtools}
\usepackage{amsfonts,amsmath,amssymb,amsthm}
\usepackage{enumerate}
\usepackage{nopageno} % usuwa numery stron
\usepackage[margin=.5in]{geometry} % zmniejsza margines
\usepackage{fouriernc} % śmieszna czcionka
\usepackage{algpseudocode}

\newcommand{\MBBR}{\mathbb{R}}
\newcommand{\MBBZ}{\mathbb{Z}}
\newcommand{\MBBN}{\mathbb{N}}
\newcommand{\MBBC}{\mathbb{C}}
\newcommand{\MBBK}{\mathbb{K}}
\newcommand{\MBBQ}{\mathbb{Q}}
\newcommand{\MCALO}{\mathcal{O}}

\newcounter{zadanie}
\newcommand{\zadanie}{\addtocounter{zadanie}{1}\section*{Zadanie \arabic{zadanie}}}

\title{Egzamin 2013/14}
\author{Stanisław Bitner}
\date{\today}

\begin{document}
\maketitle
\zadanie{}
\subsection*{(a)}
Tworzymy drzewo AVL, w którym kluczami będą pary posortowane leksykograficznie.
To pozwala na bardzo łatwe wykonywanie operacji \textit{Ini, Insert, Delete}
oraz wyszukiwania konkretnej pary. Okręgi są jednostkowe i ich środki mają
współrzędne całkowite, więc punkt $(x,y)$ może się zawierać w niewielkiej,
stałej liczbie okręgów. To oznacza, że do wykonania operacji \textit{Find}
wystarczy wyznaczyć wszystkie takie środki i sprawdzić, które z nich należą do
naszego drzewa. Jednym ze sposobów wyznaczenia nadzbioru interesujących nas
punktów jest wzięcie zbioru
$\{(a,b) | a\in[\lfloor x \rfloor - 2, \lceil x \rceil + 2] \land
           b\in[\lfloor y \rfloor - 2, \lceil y \rceil + 2]\}$.
Złożoność każdej operacji: $\MCALO(\log{n})$.

\subsection*{(b)}
Zauważmy, że z każdego środka koła można dojść w jednym ruchu do co najwyżej
$8$ innych środków. Krawędzi wychodzące z $P$ wyznaczamy po prostu przechodząc
po wszystkich środkach $S$ i dodajemy krawędź $\iff |PS| \le 1$. Dla $Q$
analogicznie. Następnie sortujemy środki po $x$ i dodajemy je do kubełków
o numerze $y$. Obsługa $y$-tego kubełka polega na utworzeniu (za pomocą
scalania) tablicy punktów składającej się z kubełków o numerach z przedziału
$[y-2,y+2]$. Iterujemy się po tak otrzymanej tablicy i dla każdego wierzchołka
sprawdzamy $20$ wierzchołków za nim. Jeśli dowolny z nich jest w odległości
$\le 2$, to tworzymy między nimi krawędź. Ostatecznie otrzymujemy graf w którym
liczba krawędzi jest rzędu $\MCALO(n)$ i na tym to grafie wykonujemy algorytm
DFS, aby sprawdzić, czy zadane punkty są w jednej spójnej.
Złożoność: $\MCALO(n + 5n + 20n) = \MCALO(n)$.

\zadanie{}
Wzbogacamy drzewo splay o następujące atrybuty:
\begin{itemize}
\item \textit{rozmiar} -- rozmiar poddrzewa;
\item \textit{leń} -- ile należy dodać do synów;
\item \textit{min} -- najmniejsza wartość w poddrzewie;
\item \textit{max} -- największa wartość w poddrzewie.
\end{itemize}
Utrzymujemy następujący niezmiennik wartość wierzchołka $v$ jest równa sumie
wartości trzymanej w wierzchołku oraz wartością \textit{leń} na jego ścieżce do
korzenia (nie licząc jego \textit{lenia}). Wszystkie atrybuty można z łatwością
utrzymywać i było to tłumaczone na ćwiczeniach bądź laboratoriach. Operację
\textit{Ini} wykonujemy poprzez wielokrotne dodanie $0$ do drzewa. Operacje
\textit{Min} i \textit{Max} wykonujemy poprzez zwrócenie odpowiednich atrybutów
korzenia. Operację \textit{Inc} wykonujemy poprzez wyekstrahowanie $k$
najmniejszych elementów za pomocą \textit{Split}. Następnie do wartości,
\textit{min, max i lenia} korzenia wyekstrahowanego drzewa dodajemy $1$. Jeśli
\textit{max} wyekstrahowanego drzewa jest większy niż \textit{min} reszty, to
używamy \textit{Split}, by wyjąć te wartości z drzew. Następnie wykonujemy
\textit{Join} łącząc drzewa w kolejności rosnących \textit{max}-ów. Należy
jeszcze pamiętać, aby wchodząc do wierzchołka najpierw wykonać leniwe
spychanie. Złożoności operacji \textit{Min} i \textit{Max}: $\MCALO(1)$.
Złożoność operacji \textit{Inc}: $\MCALO(\log{n})$.

\zadanie{}
\subsection*{(a)}
\subsection*{(b)}
\subsection*{(c)}
\subsection*{(d)}

\end{document}
