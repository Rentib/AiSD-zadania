\documentclass[12pt, a4paper]{article}
\usepackage[polish]{babel}
\usepackage[T1]{fontenc}
\usepackage[utf8]{inputenc}
\usepackage{mathtools}
\usepackage{amsfonts,amsmath,amssymb,amsthm}
\usepackage{enumerate}
\usepackage{nopageno} % usuwa numery stron
\usepackage[margin=.5in]{geometry} % zmniejsza margines
\usepackage{fouriernc} % śmieszna czcionka
\usepackage{algpseudocode}

\newcommand{\MBBR}{\mathbb{R}}
\newcommand{\MBBZ}{\mathbb{Z}}
\newcommand{\MBBN}{\mathbb{N}}
\newcommand{\MBBC}{\mathbb{C}}
\newcommand{\MBBK}{\mathbb{K}}
\newcommand{\MBBQ}{\mathbb{Q}}
\newcommand{\MCALO}{\mathcal{O}}

\newcounter{zadanie}
\newcommand{\zadanie}{\addtocounter{zadanie}{1}\section*{Zadanie \arabic{zadanie}}}

\title{Egzamin 2014/15}
\author{Stanisław Bitner}
\date{\today}

\begin{document}
\maketitle
\zadanie{}
Budujemy drzewo sufiksowe. Liściom w drzewie przypisujemy indeksy początku
słowa w nich się kończącego. Dla każdego wierzchołka programowaniem dynamicznym
wyznaczamy najmniejszy i największy indeks liścia oraz jego głębokość, czyli
sumaryczną wagę krawędzi na jego ścieżce do korzenia. Słowo reprezentowane
przez wierzchołek $v$ spełnia wymogi z treści, jeśli $max[v] - min[v] \ge
depth[v]$. Wystarczy więc znaleźć taki wierzchołek o największej głębokości
i wynikiem będzie reprezentowane przez niego słowo.
Złożoność: $\MCALO(n)$.

\zadanie{}
Wzbogacamy drzewo AVL o następujące atrybuty:
\begin{itemize}
  \item \textit{rozmiar} -- liczba elementów w poddrzewie;
  \item \textit{jedynki} -- liczba jedynek w poddrzewie;
  \item \textit{leń} -- czy należy wykonać \textit{Zamień} w synach.
\end{itemize}

Utrzymujemy następujący niezmiennik: wartość wierzchołka $v$ jest równa
alternatywie rozłącznej wartości trzymanej w wierzchołku i wszystkich wartości
\textit{leń} na jego ścieżce do korzenia (nie licząc jego \textit{lenia}).

Przy wejściu do wierzchołka $v$ lub wykonywaniu rotacji $(v,u)$ należy zepchnąć
\textit{leń} wyższego z wierzchołków, a następnie niższego (samo spychanie było
omawiane na labach). Utrzymywanie rozmiarów oraz liczby jedynek, to oczywiście
nie jest problem i wielokrotnie pojawiało się na ćwiczeniach.

Operację \textit{Zamień}, wykonujemy poprzez \textit{Split}, aby wyodrębnić elementy
z odpowiedniego przedziału. W korzeniu wyekstrahowanego poddrzewa zamieniamy
jego wartość i parametr \textit{leń} na przeciwne oraz liczbę jedynek na
$rozmiar[v] - jedynki[v]$. Na koniec wykonujemy \textit{Łącz} na otrzymanych
poddrzewach.

Operację \textit{Pozycja} wykonujemy przeszukując drzewo szukając wierzchołka
o wartości $1$ takiego, że $\mathit{jedynki}[l[v]]
+ 1 = \lfloor\frac{k}{2}\rfloor$, przy czym $k$ jest równe
$\mathit{jedynki}[root]$.

Złożoności wszystkich operacji są takie same -- $\MCALO(\log{n})$.

\zadanie{}
\subsection*{(a)}
Ukorzeniamy drzewo w $x$ i wyznaczamy $low$ dla wszystkich wierzchołków.
Wyznaczamy też dla każdego wierzchołka syna (jeśli istnieje), do którego musimy
pójść, żeby dostać się do $low[v]$. Odpowiedzią będzie cykl powstały w taki
sposób, że z $y$ prowadzimy $2$ rozłączne ścieżki: jedną do $low[y]$, drugą, do
syna $low[y]$ leżącego na ścieżce $low[y] - y$. Wielokrotnie powtarzając taki
schemat, dojdziemy w końcu do korzenia, czyli wierzchołka $x$, a tym samym
wykonamy zadanie. Złożoność: $\MCALO(n)$.

\subsection*{(b)}
Sortujemy krawędzie kubełkowo po wagach. Idziemy od najcięższych i dodajemy je
kolejno do grafu, równocześnie sprawdzając, czy $x$ i $y$ są w jednej spójnej.
Gdy dojdziemy do momentu, w którym są połączone, wykonujemy dowolny algorytm
przeszukiwania grafu i wyznaczamy dowolną ze ścieżek. Będzie to ścieżka
o największej wartości, ponieważ jeśli usuniemy krawędzie o najmniejszej wadze,
to nie będzie istniała ścieżka łącząca $x$ z $y$. Sprawdzanie, czy $x$ jest
w spójnej $y$ można robić w zamortyzowanym czasie stałym i jest to dość znany
algorytm. Złożoność: $\MCALO(n)$.

\zadanie{}
% TODO:

\end{document}
