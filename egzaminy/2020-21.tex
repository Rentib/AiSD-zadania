\documentclass[12pt, a4paper]{article}
\usepackage[polish]{babel}
\usepackage[T1]{fontenc}
\usepackage[utf8]{inputenc}
\usepackage{mathtools}
\usepackage{amsfonts,amsmath,amssymb,amsthm}
\usepackage{enumerate}
\usepackage{nopageno} % usuwa numery stron
\usepackage[margin=.5in]{geometry} % zmniejsza margines
\usepackage{fouriernc} % śmieszna czcionka
\usepackage{algpseudocode}

\newcommand{\MBBR}{\mathbb{R}}
\newcommand{\MBBZ}{\mathbb{Z}}
\newcommand{\MBBN}{\mathbb{N}}
\newcommand{\MBBC}{\mathbb{C}}
\newcommand{\MBBK}{\mathbb{K}}
\newcommand{\MBBQ}{\mathbb{Q}}
\newcommand{\MCALO}{\mathcal{O}}

\title{Egzamin 2020/21}
\author{Stanisław Bitner}
\date{\today}

\begin{document}
\maketitle

\section*{Zadanie 1}
\subsection*{(a)}
W drzewie każdy wierzchołek jest kliką, zatem największa liczba klik to $n$.
W szczególności cały graf może być kliką, więc najmniejsza liczba klik to $1$.
% TODO: zalezy od tego jak to definiujemy...

\subsection*{(b)}
W $K_n \approx n^2$, a jak $a_1 + ... + a_{100} = s$, gdzie $s$ jest ustalone,
to $\sum_{i=1}^{100} a_i^2$ jest największa, gdy jedna liczba jest bardzo duża
a pozostałe małe, natomiast najmniejsza jest, gdy wszystkie są mniej więcej
takie same.\\ Zatem największa liczba krawędzi, będzie dla $K_{2021-99}$ i $99$
krawędzi, co daje $\binom{2021}{2} + 99$.\\ Najmniejsza zaś będzie, gdy kliki
będą zbliżone rozmiarem, a zatem\\ $\sum_{i=1}^{21} \binom{21}{2}
+ \sum_{i=22}^{100} \binom{20}{2} = 21\cdot \binom{21}{2} + 79 \cdot
\binom{20}{2}$.

\subsection*{(c)}
Są dwie możliwości wyboru korzenia drzewa. Może być to punkt artykulacji lub
nie. W drugim przypadku wszystkie wybory są symetryczne.\\ Zajmijmy się
najpierw pierwszym przypadkiem -- gdy wejdziemy do którejś z klik, musimy ją
całą obejść, gdyż jedyny punkt artykulacji w grafie jest już odwiedzony
(korzeń). To oznacza, że drzewo DFS, jakie otrzymamy, będzie składało się
z korzenia i dwóch ścieżek długości $9$. To daje nam $2 \cdot (9!)^2$
możliwości DFS-numerowania.\\ W drugim przypadku jest inaczej. Zaczynamy
z dowolnego wierzchołka w grafie, dochodzimy do punktu artykulacji w $r \le 9$
ruchach. Jeśli $r=9$, to idziemy do drugiej z klik, w przeciwnym przypadku
wybieramy, do której wejdziemy. To daje nam $1 \cdot 8! \cdot 1 \cdot 9!$ dla
$r=9$ i $\sum_{r=1}^8 (r-1)! \cdot 2 \cdot (9 - r)! \cdot 9!$.\\ W sumie
dostajemy $2 \cdot (9!)^2 + 8! \cdot 9! + \sum_{r=1}^8 (r-1)! \cdot 2 \cdot (9
- r)! \cdot 9!$ możliwe DFS-numerowania.

\subsection*{(d)}
Wyznaczamy drzewo z ważonymi wierzchołkami w taki sposób, że wierzchołki to
kliki oraz punkty artykulacji. Waga punktów artykulacji to $0$ natomiast wagami
wierzchołków reprezentujących kliki są liczby ich wierzchołków. Taki graf można
łatwo otrzymać, ponieważ punkty artykulacji to punkty połączone z co najmniej
dwiema klikami. Można to obliczyć po uprzednim posortowaniu wierzchołków
kubełkowo. Odpowiedzią będzie długość średnicy w takim drzewie. Złożoność:
$\MCALO(n+k) = \MCALO(n)$.

\newpage
\section*{Zadanie 2}
Do rozwiązania tego zadania użyjemy struktury zbiorów rozłącznych.\\ Aby łatwo
odpowiadać na zapytania o zbalansowane składowe, będziemy przechowywać ich
globalny licznik $ile$ (początkowo równy $n$). Co więcej, struktury zbiorów
wzbogacamy o następujące atrybuty:
\begin{itemize}
  \item $\mathit{b}$ -- liczba białych wierzchołków w spójnej
  \item $\mathit{c}$ -- liczba czerwonych wierzchołków w spójnej
\end{itemize}
Atrybuty te należy w odpowiedni sposób aktualizować podczas funkcji
$\mathit{Kolor}$ oraz $\mathit{Dodaj}$ tak, aby licznik oraz każdy reprezentant
zbioru zawsze miał aktualne informacje. Złożoności: 
\begin{itemize}
  \item $\mathit{Kolor}$ -- $\MCALO(\alpha(n))$
  \item $\mathit{Dodaj}$ -- $\MCALO(\alpha(n))$
  \item $\mathit{ZbalansowaneSkladowe}$ -- $\MCALO(1)$
\end{itemize}
Należy przy tym pamiętać, że gdy graf stanie się spójny, wszystkie operacje
będą wykonywane w czasie stałym.

\section*{Zadanie 3}
\subsection*{(a)}
Drzewo sufiksowe i sumy prefiksowe. Złożoność: $\MCALO(n)$.

\subsection*{{(b)}}
Najpierw zajmijmy się najmniejszą sumą. Zauważmy, że nie może być ona mniejsza
niż $k-1$, gdyż jeśli pierwsza litera pojawia się w słowie $k$ razy. Można
łatwo osiągnąć taką sumę tworząc na przykład słowo $dik..ki..is..sd..d$.
Wszystkie elementy poza ostatnimi $k-1$ będą miały wartość $0$, natomiast te
ostatnie będą miały wartości $1$, do daje sumę $k-1$.\\ Największa suma będzie
na przykład dla słowa $\{diks\}^{k}$ i będzie ona równa $\frac{1 + (n-4)}{2}
\cdot (n-4) = \binom{n-3}{2}$. Pierwsze $4$ elementy tablicy będą miały
wartości $0$ natomiast kolejne będą miały wartości $1,2,..,n-4$. Nie da się więcej, bo
są $4$ różne elementy.

\section*{Zadanie 4}
Sortujemy współrzędne wierzchołków i skalujemy je do liczb $1..n$ (osobno $x$
i $y$). Relacje przecinania są w ten sposób zachowane. Następnie stosujemy
technikę zamiatania -- idziemy od lewej do prawej, gdy napotykamy lewy koniec
poziomego odcinka dodajemy jego $y$ do drzewa przedziałowego, gdy napotykamy
prawy koniec, to usuwamy. Gdy napotkamy odcinek pionowy, to liczba odcinków,
z którymi tworzy przecięcie jest równa sumie na przedziale $[y_1,y_2]$ tego
odcinka. Jeśli jest ona nie mniejsza niż dotychczasowe maksimum, to
aktualizujemy je. Analogicznie robimy od góry do dołu. Następnie tworzymy pustą
listę i powtarzamy zamiatanie, przy czym tym razem nie aktualizujemy wyniku,
tylko dodajemy numer takiego odcinka na listę. Ostatecznym wynikiem jest w ten
sposób uzyskana lista. Złożoność: $\MCALO(n\log{n})$ -- sortowanie i operacje
drzewowe powtarzane $n$-krotnie.

\end{document}
