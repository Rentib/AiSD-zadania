\documentclass[12pt, a4paper]{article}
\usepackage[polish]{babel}
\usepackage[T1]{fontenc}
\usepackage[utf8]{inputenc}
\usepackage{mathtools}
\usepackage{amsfonts,amsmath,amssymb,amsthm}
\usepackage{enumerate}
\usepackage{nopageno} % usuwa numery stron
\usepackage[margin=.5in]{geometry} % zmniejsza margines
\usepackage{fouriernc} % śmieszna czcionka
\usepackage{algpseudocode}

\newcommand{\MBBR}{\mathbb{R}}
\newcommand{\MBBZ}{\mathbb{Z}}
\newcommand{\MBBN}{\mathbb{N}}
\newcommand{\MBBC}{\mathbb{C}}
\newcommand{\MBBK}{\mathbb{K}}
\newcommand{\MBBQ}{\mathbb{Q}}
\newcommand{\MCALO}{\mathcal{O}}

\newcounter{zadanie}
\newcommand{\zadanie}{\addtocounter{zadanie}{1}\section*{Zadanie \arabic{zadanie}}}

\title{Egzamin 2018-19}
\author{Stanisław Bitner}
\date{\today}

\begin{document}
\maketitle
\zadanie{}
Do rozwiązania zadania użyjemy drzewa splay wzbogaconego o wielkości poddrzew.
Na początek sortujemy wszystkie przedziały po ich końcach. Niestety nie ma
ograniczenia na ich współrzędne, więc nie możemy tego zrobić liniowo. Następnie
idziemy po kolei po końcach przedziałów. Gdy napotkamy przedział czerwony,
dodajemy jego początek do drzewa splay. Gdy napotkamy przedział biały, dodajemy
do wyniku liczbę wartości w drzewie nie mniejszych niż początek tego
przedziału. Jako że do drzewa dodajemy tylko zakończone, czerwone przedziały,
to mamy pewność, że całkowicie się one zawierają, jeśli tylko ich początek jest
większy od początku białego przedziału. Złożoność: $\MCALO((n+m)\log{(n+m)})$.

\zadanie{}
Fakt z podstawówki: $3 \mid x \iff 3 \mid $ suma cyfr $x$.
\subsection*{(a)}
Tworzymy $3$-elementową tablicę $\mathit{pref}$, w której będziemy zapisywać
ile do tej pory było prefiksów przystających do $i$ modulo $3$. Zaczynamy
z pustym słowem, więc $pref[0] \gets 1$. Gdy przechodzimy do następnego
prefiksu, to dodajemy do globalnego licznika tyle ile do tej pory było
prefiksów przystających do obecnego prefiksu modulo $3$. Możemy tak zrobić,
ponieważ $x[1..j] \equiv_3 x[1..i] \implies x[j+1..i] \equiv_3 0$.\\
Złożoność: $\MCALO(|x|)$

\begin{algorithmic}

\Function{Podzielne}{x}
  \State $pref[3] \gets {1,0,0}, curr \gets 0, res \gets 0$
  \For{$i := 1..n$}
    \State $curr \gets (curr + x[i]) mod 3$
    \State $res \gets res + pref[curr]$
    \State $pref[curr] \gets pref[curr] + 1$
  \EndFor
\EndFunction

\end{algorithmic}

\subsection*{(b)}
Postępujemy podobnie jak ostatnio, ale zapamiętujemy najmniejszy i największy
indeks dla każdego $i\in\{0,1,2\}$ (przystawanie modulo $3$). Otrzymujemy
w ten sposób pary $(i,j)$ oznaczające, że $x[i+1..j] \equiv_3 0$. Z par trzeba
usunąć jeszcze zera wiodące, co robimy liniowo. Wynikiem będzie para
o największej różnicy $j-i$. W przypadku remisów porównujemy te słowa
leksykograficznie (będą co najwyżej $2$ porównania. Złożoność: $\MCALO(|x|)$.

\subsection*{(c)}

\zadanie{}
Strukturą, jakiej użyjemy będzie drzewo AVL wzbogacone o następujące atrybuty:
\begin{itemize}
  \item $bmax,cmax$ -- największy element dla obu kolorów;
  \item $bmin,cmin$ -- najmniejszy element dla obu kolorów;
  \item $res$ -- wynik dla poddrzewa.
\end{itemize}
Dla każdego wierzchołka $max, min$ można trywialnie obliczać na podstawie jego
wartości i wartości w jego synach. $res$ można obliczać biorąc najmniejszą
wartość spośród $l.res, r.res$ oraz różnice $r.bmin - w, w - l.bmax$, jeśli
wierzchołek jest czerwony, lub $r.cmin - w, w - l.cmax$, jeśli jest biały ($w$
to wartość wierzchołka). Jako że atrybuty można utrzymywać na podstawie
atrybutów dzieci, to nie ma problemu z rotacjami.
Złożoności:
\begin{itemize}
  \item $\mathit{Insert}$ -- $\MCALO(\log{n})$
  \item $\mathit{Delete}$ -- $\MCALO(\log{n})$
  \item $\mathit{Gap}$ -- $\MCALO(1)$ -- czytamy wynik korzenia
\end{itemize}

\zadanie{}
\subsection*{(a)}
Niech $\forall_{i\in{1..n+2}} a[i] = 1$, wtedy $G_a = K_n$, zatem wysokość drzewa
DFS, będzie równa $n-1$. Nie da się otrzymać większej, ponieważ jest to
największa wysokość dla $n-drzewa$.

\subsection*{(b)}
Zauważmy, że $\forall_{i\in{1..n}} \{i,i+1\}, \{i,i+2\} \in E$. Niech $a[i]
= i$, wtedy istnieją tylko krawędzie, które muszą istnieć. Jeśli dodamy
jakąkolwiek krawędź, to największa odległość $1$ od wierzchołka może tylko
zmaleć, więc wysokość też może tylko zmaleć. Wysokość drzewa BFS będzie równa
$\lfloor\frac{n}{2}\rfloor$.

\subsection*{(c)}
Należy zauważyć, że jeśli graf zawierający jedynie te krawędzie, które muszą
być w nim zawarte jest dwuspójny, to gdy dodamy jakiekolwiek inne krawędzie,
pozostanie dwuspójny. Wystarczy więc udowodnić, że graf
$\langle V=\{1..n\},E = \{ \{i,i+1\} , \{i,i+2\} \}\rangle$
jest dwuspójny. Przypuśćmy, że listy sąsiedztwa są posortowane. Wtedy
zaczynając DFS z wierzchołka numer $1$, będzie miał on jednego syna oraz
$\forall_i parent[i] = i-1 \land low[i] \le i-2$, zatem graf jest dwuspójny.

\subsection*{(d)}
Tworzymy $n$ kubełków. Przechodzimy tablicę $a$ i wrzucamy indeksy do
odpowiednich kubełków. Indeks $j$ wrzucamy do $\{a[j],a[j+1],a[j+2]\}$-tego
kubełka. Następnie przeglądamy kolejne kubełki.\\
W $k$-tym kubełku, jeśli $2$ kolejne elementy nie są w jednym zbiorze, to
łączymy je i dodajemy $k$ do wagi. Wewnątrz każdego z kubełków istnieją kliki,
jednak wystarczy, że rozpatrzymy jedynie pojedynczą ścieżkę rozpinającą, gdyż
wszystkie krawędzie wewnątrz kubełka mają jednakowe wagi. To, że otrzymamy
minimalne drzewo rozpinające wynika z algorytmu Kruskala.

\end{document}
