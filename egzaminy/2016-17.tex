\documentclass[12pt, a4paper]{article}
\usepackage[polish]{babel}
\usepackage[T1]{fontenc}
\usepackage[utf8]{inputenc}
\usepackage{mathtools}
\usepackage{amsfonts,amsmath,amssymb,amsthm}
\usepackage{enumerate}
\usepackage{nopageno} % usuwa numery stron
\usepackage[margin=.5in]{geometry} % zmniejsza margines
\usepackage{fouriernc} % śmieszna czcionka
\usepackage{algpseudocode}

\newcommand{\MBBR}{\mathbb{R}}
\newcommand{\MBBZ}{\mathbb{Z}}
\newcommand{\MBBN}{\mathbb{N}}
\newcommand{\MBBC}{\mathbb{C}}
\newcommand{\MBBK}{\mathbb{K}}
\newcommand{\MBBQ}{\mathbb{Q}}
\newcommand{\MCALO}{\mathcal{O}}

\newcounter{zadanie}
\newcommand{\zadanie}{\addtocounter{zadanie}{1}\section*{Zadanie \arabic{zadanie}}}

\title{Egzamin 2016/17}
\author{Stanisław Bitner}
\date{\today}

\begin{document}
\maketitle
\zadanie{}
\subsection*{(a)}
Będzie tylko jedno drzewo BFS. W każdym ruchu odwiedzamy $2$ kolejne
wierzchołki. Z każdego z nich wychodzi po jednej krawędzi, do \textbf{różnych}
nieodwiedzonych jeszcze wierzchołków. Można udowodnić indukcyjnie, ale chyba
nie ma potrzeby.

\subsection*{(b)}
Niech $D_n$ będzie liczbą różnych drzew DFS dla $L_n$. Wtedy
$D_n = D_{n-1} + D_{n-2} + D_{n-3}$.\\ $D = (1,1,2,4,7,13,24,44,81,149...)
\implies D_{10} = 149$. Równanie rekurencyjne tak wygląda, ponieważ możemy albo
skoczyć o $1$ i wtedy jest $D_{n-1}$, albo o $2$ i następnie w lewo
($D_{n-3}$), lub w prawo ($D_{n-2}$).

\subsection*{(c)}
Niech $d_k$ będzie wagą MST obejmującego wierzchołki $1..k$, wtedy $d_1 = 0,
d_2 = w(1,2)$,
$$
d_k = \min\begin{cases}
        d_{k-1} + w(k,k-1)\\
        d_{k-1} + w(k,k-2)\\
        d_{k-2} + w(k,k-1) + w(k,k-2)\\
      \end{cases}
.
$$
Dla kolejnych $k$ od $2$ do $n$ wyznaczamy $d_k$. Waga MST będzie równa $d_k$.
Aby wyznaczyć, które krawędzie wchodzą w skład MST idziemy od tyłu i bierzemy
krawędzie, których waga została dodana w najlepszym wyniku.

\subsection*{(d)}
% TODO:

\zadanie{}
% TODO:

\zadanie{}
Tworzymy drzewo sufiksowe dla słowa $x \cdot x \cdot \# \cdot y \cdot \$$.
Każdemu liściowi przypisujemy indeks początku reprezentowanego przez niego
podsłowa. Dynamicznie obliczamy minimum i maksimum tych indeksów w poddrzewach
wierzchołków. Wykonujemy algorytm DFS, gdy znajdziemy wierzchołek o głębokości
-- liczonej w sumarycznej długości krawędzi -- nie mniejszej niż $|x|$ oraz
taki, że $min \le 2 \cdot |x| < max$, to dodajemy $1$ do globalnego wyniku
i nie schodzimy już do jego synów.
Złożoność: $\MCALO(n)$.

\end{document}
