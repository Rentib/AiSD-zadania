\documentclass[12pt, a4paper]{article}
\usepackage[polish]{babel}
\usepackage[T1]{fontenc}
\usepackage[utf8]{inputenc}
\usepackage{mathtools}
\usepackage{amsfonts,amsmath,amssymb,amsthm}
\usepackage{enumerate}
\usepackage{nopageno} % usuwa numery stron
\usepackage[margin=.5in]{geometry} % zmniejsza margines
\usepackage{fouriernc} % śmieszna czcionka
\usepackage{algpseudocode}

\newcommand{\MBBR}{\mathbb{R}}
\newcommand{\MBBZ}{\mathbb{Z}}
\newcommand{\MBBN}{\mathbb{N}}
\newcommand{\MBBC}{\mathbb{C}}
\newcommand{\MBBK}{\mathbb{K}}
\newcommand{\MBBQ}{\mathbb{Q}}
\newcommand{\MCALO}{\mathcal{O}}

\newcounter{zadanie}
\newcommand{\zadanie}{\addtocounter{zadanie}{1}\section*{Zadanie \arabic{zadanie}}}

\title{Egzamin 2016/17}
\author{Stanisław Bitner}
\date{\today}

\begin{document}
\maketitle
\zadanie{}
\subsection*{(a)}
Będzie tylko jedno drzewo BFS. W każdym ruchu odwiedzamy $2$ kolejne
wierzchołki. Z każdego z nich wychodzi po jednej krawędzi, do \textbf{różnych}
nieodwiedzonych jeszcze wierzchołków. Można udowodnić indukcyjnie, ale chyba
nie ma potrzeby.

\subsection*{(b)}
Niech $D_n$ będzie liczbą różnych drzew DFS dla $L_n$. Wtedy
$D_n = D_{n-1} + D_{n-2} + D_{n-3}$.\\ $D = (1,1,2,4,7,13,24,44,81,149...)
\implies D_{10} = 149$. Równanie rekurencyjne tak wygląda, ponieważ możemy albo
skoczyć o $1$ i wtedy jest $D_{n-1}$, albo o $2$ i następnie w lewo
($D_{n-3}$), lub w prawo ($D_{n-2}$).

\subsection*{(c)}
Robimy dynamika po wierzchołkach:
% TODO:

\subsection*{(d)}
% TODO:

\zadanie{}

\zadanie{}
Konstruujemy \textit{ROT} dla $x$ i \textit{SUF} dla $y$, a także wyznaczamy
KMR na słowie $x \cdot y$. Następnie dla każdego różnego okresu $x$
reprezentowanego jako indeks $i$ ($c_i = x[i..n] \cdot x[1..i-1]$) szukamy
binarnie pierwszego leksykograficznie większego sufiksu $y$. Słowa można
porównywać za pomocą struktury KMR w czasie stałym. Jeżeli pierwsze $|x|$
elementów tego sufiksu jest równe $x$, to za pomocą na przykład statycznego
drzewa przedziałowego wyznaczamy najdłuższy spójny podciąg sufiksów na prawo od
znalezionego, których $lcp \ge |x|$ i dodajemy jego długość do wyniku.
Złożoność: $\MCALO((|x|+|y|)\log{(|x|+|y|)})$.
% TODO:

\end{document}
