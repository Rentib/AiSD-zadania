\documentclass[12pt, a4paper]{article}
\usepackage[polish]{babel}
\usepackage[T1]{fontenc}
\usepackage[utf8]{inputenc}
\usepackage{mathtools}
\usepackage{amsfonts,amsmath,amssymb,amsthm}
\usepackage{enumerate}
\usepackage{nopageno} % usuwa numery stron
\usepackage[margin=.5in]{geometry} % zmniejsza margines
\usepackage{fouriernc} % śmieszna czcionka
\usepackage{algpseudocode}

\newcommand{\MBBR}{\mathbb{R}}
\newcommand{\MBBZ}{\mathbb{Z}}
\newcommand{\MBBN}{\mathbb{N}}
\newcommand{\MBBC}{\mathbb{C}}
\newcommand{\MBBK}{\mathbb{K}}
\newcommand{\MBBQ}{\mathbb{Q}}
\newcommand{\MCALO}{\mathcal{O}}

\newcounter{zadanie}
\newcommand{\zadanie}{\addtocounter{zadanie}{1}\section*{Zadanie \arabic{zadanie}}}

\title{Egzamin 2015/16}
\author{Stanisław Bitner}
\date{\today}

\begin{document}
\maketitle
\zadanie{}

Do rozwiązania następujących podpunktów przyda nam się znajomość stopni wejścia
i wyjścia wszystkich wierzchołków. Można je wyliczyć w czasie liniowym.
Sortujemy przedziały wierzchołków kubełkowo po wartościach $l$. Przechodzimy
liniowo od lewej do prawej -- stopień wejścia wierzchołka $i$ jest równy
liczbie zaczętych i niezakończonych jeszcze przedziałów. Stopień wyjścia
wierzchołka $v$ to $r[v] - l[v] + 1$. Odejmujemy jeszcze $1$ od każdego
wierzchołka z pętlą ($l[v] \le v \le r[v]$).
Złożoność: $\MCALO(n)$.

\subsection*{(a)}
Sprawdzamy, czy istnieje dokładnie $1$ wierzchołek o stopniu wejścia równym $0$
i czy istnieje dokładnie $n-1$ krawędzi. Jeśli tak, to tworzymy listę
sąsiedztwa, odpalamy $dfs$ z wierzchołka o stopniu wejścia równym $0$ i jeśli
odwiedzimy wszystkie wierzchołki, to odpowiedzią jest \textit{TAK}.
W przeciwnym razie odpowiedzią jest \textit{NIE}. Warunki te są konieczne
i wystarczające z definicji drzewa (spójny graf o $n-1$ krawędziach).
Złożoność: $\MCALO(n)$.

\subsection*{(b)}
Sortujemy przedziały i od lewej do prawej scalamy $2$ kolejne, jeśli nie są
rozłączne. Znowu idziemy od lewej do prawej i dla wierzchołka $i$ wykonujemy
$3$ operacje \textit{Union}. Z $l[i], r[i]$ oraz początkiem przedziału, do
którego należy $i$-ty wierzchołek po scalaniu, jeśli taki przedział istnieje.
Sprawdzamy, czy na końcu wszystkie wierzchołki są w jednym zbiorze i jeśli tak
jest, to graf jest słabo spójny.\\
Złożoność: $\MCALO(n\alpha(n))$.

\subsection*{(c)}
Sprawdzamy, czy graf jest słabo spójny oraz, czy sumaryczna liczba stopni
wyjścia jest równa sumarycznej liczbie stopni wejścia. Jeśli tak jest, to graf
jest eulerowski, w przeciwnym razie nie jest. Wiemy, że warunki te są konieczne
i wystarczające z wykładu z matematyki dyskretnej.
Złożoność: $\MCALO(n\alpha(n))$.

\zadanie{}
\subsection*{(a)}
\begin{algorithmic}
\State $p \gets 0$
\For{$i:=1..n-1$}
  \If{$a[i] > a[i+1]$}
    \State $swap(a[i], a[i+1])$
    \If{$p=1$}
      \State \Return
    \EndIf
    \State $p \gets 1$
  \EndIf
\EndFor
\end{algorithmic}
Wykonamy co najwyżej $n-1 = 2015$ porównań, co jest liczbą minimalną, bo może
być $0$ inwersji i wtedy trzeba sprawdzić, że rzeczywiście cały ciąg jest
posortowany.

\subsection*{(b)}
% TODO:

\zadanie{}
Tworzymy drzewo sufiksowe dla słowa $x \cdot x \cdot \$ \cdot y \cdot \#$.
Następnie szukamy wierzchołka o głębokości $|x|$, który jest przodkiem $\$$
oraz $\#$. Jeśli taki wierzchołek istnieje, to w $y$ znajduje się pewne
przesunięcie cykliczne $x$. Głębokość definiujemy jako sumaryczną wagę krawędzi
na ścieżce z korzenia do wierzchołka.

\end{document}
