\documentclass[12pt, a4paper]{article}
\usepackage[polish]{babel}
\usepackage[T1]{fontenc}
\usepackage[utf8]{inputenc}
\usepackage{mathtools}
\usepackage{amsfonts,amsmath,amssymb,amsthm}
\usepackage{enumerate}
\usepackage{nopageno} % usuwa numery stron
\usepackage[margin=.5in]{geometry} % zmniejsza margines
\usepackage{fouriernc} % śmieszna czcionka
\usepackage{listing}
\usepackage{algpseudocode}

\newcommand{\MBBR}{\mathbb{R}}
\newcommand{\MBBZ}{\mathbb{Z}}
\newcommand{\MBBN}{\mathbb{N}}
\newcommand{\MBBC}{\mathbb{C}}
\newcommand{\MBBK}{\mathbb{K}}
\newcommand{\MBBQ}{\mathbb{Q}}
\newcommand{\MCALO}{\mathcal{O}}

\title{Egzamin 2021/22}
\author{Stanisław Bitner}
\date{\today}

\begin{document}
\maketitle

\section*{Zadanie 1}
Dla wygody przypuśćmy, że liczba jest postaci $\sum_{i=1}^n s_i \cdot 2^i$, czyli że
bity mniej znaczące są po lewej stronie. Jeśli nie może tak być, to wystarczy odwrócić cały napis.
\subsection*{(a)}
Zauważmy, że
\begin{align*}
  3 \mid s[k..i] \iff 3 \mid 2^{k-1} \cdot s[k..i] \iff 3 \mid s[1..i] - s[1..k-1]
  \iff s[1..i] \equiv_3 s[1..k-1].
\end{align*}
Z tego wynika, że aby zliczyć liczbę podsłów podzielnych przez 3, kończących się w $i$ wystarczy
obliczyć liczbę prefiksów $s[1..j]$ słowa $s[1..i]$ takich, że $s[1..j] \equiv_3 s[1..i]$.
Warto również zauważyć, że $2^k\equiv_3 (-1)^k$, to pozwala uniknąć operacji na dużych liczbach.
Złożoność: $\MCALO(n)$.

\begin{algorithmic}
\State $res \gets 0, m \gets 1, sum \gets 0, p[3] \gets \{1,0,0\}$
\For{$i := 1..n$}
  \State $sum \gets (sum + s[i] \cdot m) \mod 3$
  \State $res \gets res + p[sum]$
  \State $p[sum] \gets p[sum] + 1$
  \State $m \gets -m$
\EndFor
\end{algorithmic}

\subsection*{(b)}
Do rozwiązania użyjemy wzbogaconego drzewa splay.
Wstawianie i usuwanie elementów można zrobić, dodając do wierzchołków informacje
o rozmiarze ich poddrzew. Do wierzchołków dodamy również informacje o liczbie podsłów
podzielnych przez 3 w ich poddrzewach oraz krotnościach właściwych prefiksów i sufiksów przystających
do $0,1,2 \mod 3$.\\
Zawartość wierzchołka --
$\{$ klucz, rozmiar (poddrzewa), wynik, pre[3], suf[3] $\}$.
Przy tak zdefiniowanym drzewie pierwsze 2 polecenia wykonujemy w złożoności
$\MCALO(\log{n})$, natomiast odpowiedzi na zapytanie \textit{Podzielne3} mamy
w $\MCALO(1)$.\\ Wartości $pre, suf$ i $wynik$ można wyliczać na podstawie
klucza w wierzchołku i wartości w jego synach, więc nie ma problemu
z rotacjami. Należy przy tym pamiętać, że obliczając wartości, może zmienić się
parzystość indeksów prawego poddrzewa; należy wówczas zamieniać wartości
przystające do $1$ z wartościami przystającymi do $-1$ ($2^n\equiv_3 -1$).

\subsection*{(c)}
Liczba jest podzielna przez $4$, gdy jej pierwsze $2$ bity to zera
(założyliśmy, że najmniej znaczące są z lewej strony). Wystarczy wyliczyć
liczbę różnych podsłów $00\{1,0\}^*1$. Konstruujemy drzewo sufiksowe w taki
sposób, że wagi krawędzi utożsamiamy z liczbą jedynek. Odpowiedzią jest suma
wag krawędzi poddrzewa słowa $00$.
Złożoność: $\MCALO(n)$.

\section*{Zadanie 2}
\subsection*{(a)}
Tak samo, jak kolejny podpunkt, ale z założeniem, że $c=0$.
Innym sposobem jest zmiana jednej wartości w $1$, a drugiej w $-1$ i użycie
dość znanego algorytmu używającego sum prefiksowych do wyznaczania najdłuższego
spójnego podciągu o sumie równej $0$.
Złożoność: $\MCALO(n)$.

\subsection*{(b)}
Każdy prefiks takiego ciągu reprezentujemy jako krotkę $(a, b, c)$, mającą
w sobie krotności kolejnych liczb. Następnie każdy prefiks utożsamiamy z krotką
$k_i = (a-b, a-c, i)$, gdzie $i$ to długość prefiksu. Sortujemy te krotki
leksykograficznie, będzie to liniowe, gdyż $a-b, a-c, i \in [-n, n]$. Teraz
w każdej klasie krotek, wyznaczonej przez $2$ pierwsze elementy, szukamy
najmniejszego i największego indeksu (wystarczy liniowo, bo i tak każdą krotkę
sprawdzimy raz). Wynikiem będzie para ($i_{min} + 1, j_{max}$) taka, że $i$ oraz
$j$ są z jednej klasy i ich różnica jest największa.
Złożoność: $\MCALO(n)$.

\section*{Zadanie 3}
Użyjemy struktury zbiorów rozłącznych. Zaczynamy z singletonami. Każdy
wierzchołek będzie przechowywał swój stopień, natomiast każdego reprezentanta
w strukturze wzbogacamy o następujące atrybuty:
\begin{itemize}
  \item \textit{p} -- liczba wierzchołków o parzystych stopniach;
  \item \textit{n} -- liczba wierzchołków o nieparzystych stopniach.
\end{itemize}
Atrybuty te można bardzo łatwo utrzymywać podczas dodawania krawędzi. Dodatkowo
trzymamy globalny licznik \textit{e} oznaczający liczbę takich spójnych, że
wszystkie ich wierzchołki mają parzyste stopnie. Oczywiście parzystość stopni
i spójność to warunki konieczne i wystarczające, aby w grafie był cykl eulera.
Złożoności:
\textit{Dodaj} -- $\MCALO(\alpha(n))$,
\textit{EulerCyc} -- $\MCALO(1)$.

\section*{Zadanie 4}
Jako że dodajemy coraz to większe klucze, to interesuje nas jedynie skrajnie
prawa ścieżka. Jeśli istnieje na niej jakiś wierzchołek $v$ taki, że $B(v)=-1$
i dla każdego potomka $v, B(v) = 0$, to po operacji $\mathit{Insert}$ musimy
wykonać rotacje na wysokości $h(v)$. Po wykonaniu jej wszystkie wierzchołki,
które były potomkami $v$ leżącymi na skrajnie prawej ścieżce, będą miały
$B(u)=-1$. Zajmijmy się sumarycznym kosztem przejścia z pełnego drzewa
binarnego o wysokości $h$, z jednym skrajnie prawym liściem, do pełnego drzewa
binarnego o wysokości $h+1$. Oczywiście wykonamy $I_h = 2^h - 1$ operacji
$\mathit{Insert}$. Sumaryczny koszt można wyrazić funkcją rekurencyjną $A_h
= A_{h-1} + h + A_{h-1} = 2A_{h-1} + h \approx 3\cdot 2^h$. $\frac{A_h}{I_h}
\approx 3$, zatem operacja $\mathit{Insert}$ ma średnią (zamortyzowaną)
złożoność $\MCALO(1)$.

\end{document}
