\documentclass[12pt, a4paper]{article}
\usepackage[polish]{babel}
\usepackage[T1]{fontenc}
\usepackage[utf8]{inputenc}
\usepackage{mathtools}
\usepackage{amsfonts,amsmath,amssymb,amsthm}
\usepackage{enumerate}
\usepackage{nopageno} % usuwa numery stron
\usepackage[margin=.5in]{geometry} % zmniejsza margines
\usepackage{fouriernc} % śmieszna czcionka
\usepackage{listing}
\usepackage{algpseudocode}

\newcommand{\MBBR}{\mathbb{R}}
\newcommand{\MBBZ}{\mathbb{Z}}
\newcommand{\MBBN}{\mathbb{N}}
\newcommand{\MBBC}{\mathbb{C}}
\newcommand{\MBBK}{\mathbb{K}}
\newcommand{\MBBQ}{\mathbb{Q}}
\newcommand{\MCALO}{\mathcal{O}}

\title{Egzamin 2021/22}
\author{Stanisław Bitner}
\date{\today}

\begin{document}
\maketitle

\section*{Zadanie 1}
Dla wygody przypuśćmy, że liczba jest postaci $\sum_{i=1}^n s_i \cdot 2^i$, czyli że
bity mniej znaczące są po lewej stronie. Jeśli nie może tak być, to wystarczy odwrócić cały napis.
\subsection*{(a)}
Zauważmy, że
\begin{align*}
  3 \mid s[k..i] \iff 3 \mid 2^{k-1} \cdot s[k..i] \iff 3 \mid s[1..i] - s[1..k-1]
  \iff s[1..i] \equiv_3 s[1..k-1].
\end{align*}
Z tego wynika, że aby zliczyć liczbę podsłów podzielnych przez 3, kończących się w $i$ wystarczy
obliczyć liczbę prefiksów $s[1..j]$ słowa $s[1..i]$ takich, że $s[1..j] \equiv_3 s[1..i]$.
Warto również zauważyć, że $2^k\equiv_3 (-1)^k$, to pozwala uniknąć operacji na dużych liczbach.
Złożoność: $\MCALO(n)$.

\begin{algorithmic}
\State $res \gets 0, m \gets 1, sum \gets 0, p[3] \gets \{1,0,0\}$
\For{$i := 1..n$}
  \State $sum \gets (sum + s[i] \cdot m) \mod 3$
  \State $res \gets res + p[sum]$
  \State $p[sum] \gets p[sum] + 1$
  \State $m \gets -m$
\EndFor
\end{algorithmic}

\subsection*{(b)}
Do rozwiązania użyjemy wzbogaconego drzewa splay.
Wstawianie i usuwanie elementów można zrobić, dodając do wierzchołków informacje
o rozmiarze ich poddrzew. Do wierzchołków dodamy również informacje o liczbie podsłów
podzielnych przez 3 w ich poddrzewach oraz krotnościach właściwych prefiksów i sufiksów przystających
do $0,1,2 \mod 3$.\\
Zawartość wierzchołka --
$\{$ klucz, rozmiar (poddrzewa), wynik, pre[3], suf[3] $\}$.
Przy tak zdefiniowanym drzewie pierwsze 2 polecenia wykonujemy w złożoności
$\MCALO(\log{n})$, natomiast odpowiedzi na zapytanie \textit{Podzielne3} mamy
w $\MCALO(1)$.\\ Wartości $pre, suf$ i $wynik$ można wyliczać na podstawie
klucza w wierzchołku i wartości w jego synach, więc nie ma problemu
z rotacjami. Należy przy tym pamiętać, że obliczając wartości, może zmienić się
parzystość indeksów prawego poddrzewa; wówczas należy zamieniać wartości
przystające z $1$ z wartościami przystającymi do $-1$ ($2^n\equiv_3 -1$).

\subsection*{(c)}
Liczba jest podzielna przez $4$, gdy jej pierwsze $2$ bity to zera
(założyliśmy, że najmniej znaczące są z lewej strony).
Wystarczy wyliczyć liczbę różnych podsłów $00\{1,0\}^*1$.
Wyliczamy drzewo sufiksowe dla słowa $s$ -- odpowiedzią będzie liczba krawędzi w poddrzewie $00$.
Złożoność: $\MCALO(n)$.

\section*{Zadanie 2}
\subsection*{(a)}
Wyznaczamy te dwie różne liczby (liniowo przechodząc po tablicy i na przykład porównując dwie sąsiednie).
Mapujemy jedną do $1$ drugą do $-1$. Teraz naszym zadaniem jest znaleźć najdłuższy spójny podciąg
o sumie $0$. Zauważmy, że suma każdego podciągu jest nie większa niż $n$. Możemy więc zapisać
dla każdego $k \in \{1..n\}$ najmniejszy oraz największy indeks $i$ taki, że $\sum_{i}^k mapped(a[i]) = k$.
Następnie wynikiem będzie para indeksów przypisanych sumie $k$ o największej różnicy.
Przy czym taki algorytm zwraca parę $(i-1, j)$, więc w ostatecznym wyniku należy pamiętać o zwiększeniu
o $1$ mniejszego z indeksów.
Złożoność: $\MCALO(n)$.

\subsection*{(b)}
Każdy prefiks takiego ciągu reprezentujemy jako krotkę $(a, b, c)$, mającą
w sobie krotności kolejnych liczb. Następnie każdy prefiks utożsamiamy z krotką
$k_i = (a-b, a-c, i)$, gdzie $i$ to długość prefiksu. Sortujemy te krotki
leksykograficznie, będzie to liniowe, gdyż $a-b, a-c, i \in [-n, n]$. Teraz
w każdej klasie krotek, wyznaczonej przez $2$ pierwsze elementy, szukamy
najmniejszego i największego indeksu (wystarczy liniowo, bo i tak każdą krotkę
sprawdzimy raz). Wynikiem będzie para ($i_min + 1, j_max$) taka, że $i$ oraz
$j$ są z jednej klasy i ich różnica jest największa.

\section*{Zadanie 3}

Użyjemy struktury zbiorów rozłącznych. Spójny graf posiada oczywiście cykl Eulera $\iff$ stopień
każdego wierzchołka jest parzysty. Zaczynamy z lasem drzew jedno-wierzchołkowych i globalnym licznikiem
spójnych Eulerowskich ustawionym na $n$. EulerCyc będzie po prostu zwracał ten licznik.
\begin{algorithmic}
\State $euler \gets n, deg[n] \gets {0,..,0}, dsudeg[n] \gets {0,..,0}$
\Function{Dodaj}{$u,v$}
\State $pu \gets Find(u)$
\State $pv \gets Find(v)$
\State $euler \gets euler - [dsudeg[pu][1] = 0] - [dsudeg[pv][1] = 0]$
\State $dsudeg[pu][deg[u]] -= 1$
\State $dsudeg[pv][deg[v]] -= 1$
\State $deg[u] \gets 1 - deg[u]$
\State $deg[v] \gets 1 - deg[v]$
\State $dsudeg[pu][deg[u]] += 1$
\State $dsudeg[pv][deg[v]] += 1$
\State $Union(u, v)$
\State $rep \gets Find(v)$
\State $dsudeg[rep] = dsudeg[pu] + dsudeg[pv]$
\State $euler \gets euler + [dsudeg[rep][1] = 0]$
\EndFunction
\end{algorithmic}
Dla każdego zbioru podtrzymujemy jego liczbę wierzchołków o stopniach parzystych oraz nieparzystych.
Żeby było łatwiej jako stopień wierzchołka trzymamy tylko wynik modulo $2$.
Złożoność: $\MCALO(n\alpha(n))$.

\section*{Zadanie 4}

\end{document}
